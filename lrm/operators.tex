\section{Operators and Precedences}%
\label{sec:operators_and_precedences}

The operators on integer, double, boolean, and tensor literal types
supported by TensorFlock are:

\begin{center}
\begin{tabular}{ |c|c|c|c| } 
 \hline
 Precedence & Operator Class & Operators & Associativity\\ 
 \hline
 1 & Tensor Indexing & \verb|[]| & Left to Right\\
 \hline
 2 & Unary Operators & \verb|-,!| & Right to Left \\
 \hline
 3 & Exponentiation & \verb|^| & Left to Right\\
 \hline
 4 & Multiplicative Operators & \verb|*,/,%| & Left to Right\\
 \hline
 5 & Additive Binary Operators & \verb|+,-| & Left to Right\\
 \hline
 6 & Relational & \verb|<,>,<=,>=| & Left to Right\\
 \hline
 7 & Equality & \verb|==,!=| & Left to Right\\
 \hline
 8 & Logical And & \verb|&&| & Left to Right\\
 \hline
 9 & Logical Or & \verb!||! & Left to Right\\
 \hline
 10 & Assignment & \verb|=| & Right to Left\\
 \hline
\end{tabular}
\end{center}

\subsection{Tensor Operators}
The following operators can operate on tensors in a per-component manner:
\begin{itemize}
    \item Unary Operators
    \item Exponentiation (to a scalar)
    \item Multiplicative Operators
    \item Additive Binary Operators
    \item Relational 
    \item Equality
    \item Logical And
    \item Logical Or
\end{itemize}
For example, the following function declaration is well defined:
\begin{lstlisting}
scalar_mult : Double -> T<a,b> -> T<a,b>
scalar_mult n t = n * t
\end{lstlisting}
This will return a new tensor of the same shape with each component multiplied
by n. Likewise, equality comparison is calculated componentwise. Relational
operations, logical and, and logical or  on \verb|T<n,m> OP T<n,m>| return a 
new tensor of 1's and 0's by comparison of components.
\\
For any of these operations between tensors, correct tensor shape is immediately checked to
determine whether the operation is allowed.

\subsection{Scope and Definition}
Function definition in a function declaration, must have an
identifier followed by zero or more formal arguments on the left-hand side. On
the right-hand side, the expression given may be followed by a scope of locally
declared variables. For example:
\begin{lstlisting}
addDouble : Double; 
addDouble = a + b; 
{ 
    a : Double; a = 3.14;
    b : Double; b = 6.28;
}
\end{lstlisting}
Definition that occurs in the declaration scope such that the operation 
on the variables returns the correct type as specified by the function header
is well defined.
