\section{Operators and Precedences}%
\label{sec:operators_and_precedences}

The operators on \verb|Int|, \verb|Bool|, and \verb|T<>| types
supported by TensorFlock are:

\begin{table}[htpb]
    \centering
    \caption{Operator precedences}
    \label{tab:label}
\begin{tabular}{ |c|c|c|c| } 
 \hline
 Precedence & Operator Class & Operators & Associativity\\ 
 \hline
 1 & Tensor Indexing & \verb|[]| & Left to Right\\
 \hline
 2 & Function Application & \{\textit{whitespace}\} & Left to Right \\
 \hline
 3 & Exponentiation & \verb|^| & Left to Right\\
 \hline
 4 & Unary Operators & \verb|-| & Right to Left \\
 \hline
 5 & Multiplicative Operators & \verb|*,/,%| & Left to Right\\
 \hline
 6 & Additive Binary Operators & \verb|+,-| & Left to Right\\
 \hline
 7 & Relational & \verb|<,>,<=,>=| & Left to Right\\
 \hline
 8 & Equality & \verb|==,!=| & Left to Right\\
 \hline
 9 & Logical And & \verb|&&| & Left to Right\\
 \hline
 10 & Logical Or & \verb!||! & Left to Right\\
 \hline
 11 & Conditional & \verb!if _ then _ else _! & Left to Right\\
 \hline
 12 & Definition & \verb|=| & Right to Left\\
 \hline
\end{tabular}
    
\end{table}
\begin{center}
\end{center}

\subsection{Tensor Operators}
For tensors of rank $> 0$, only the arithmetic operators are defined. All
operators returning \verb|Bool| do not type check when applied to objects of
type \verb|T|.
For example, the following function declaration is well defined:
\begin{lstlisting}
scalar_mult : T<> -> T<a,b> -> T<a,b>;
scalar_mult n t = n * t[i,j];
\end{lstlisting}
This will return a new tensor of the same shape with each component multiplied
by n. Likewise, equality comparison is calculated componentwise. Relational
operations, logical and, and logical on tensors are only defined on rank
0 tensors.  For any of these operations between tensors, correct tensor shape is
compile-time checked to determine whether the operation is allowed.

\subsection{Scope and Definition}
Function definition in a function declaration, must have an
identifier followed by zero or more formal arguments on the left-hand side. On
the right-hand side, the expression given may be followed by a scope of locally
declared variables. For example:
\begin{lstlisting}
addDouble : T<>; 
addDouble = a + b; 
{ 
    a : T<>; a = 3.14;
    b : T<>; b = 6.28;
}
\end{lstlisting}
Definition that occurs in the declaration scope such that the operation 
on the variables returns the correct type as specified by the function header
is well defined.
