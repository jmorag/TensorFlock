\section{Tensor Indexing}%
\label{sec:tensor_indexing}
Tensor indexing is evaluated as follows:
\begin{itemize}
    \item When an index appears in a bracket, bind immediately to that shape.
    \item In the case of an unbound index, defer binding until 
        innermost scope has been explored.
\end{itemize}

\begin{lstlisting}[title={Fill a rank 1 tensor with entries 1 to n}]
one_through_n : Int -> T<n>;
one_through_n n = t; 
    { t : T<n>; 
      t[i] = cast(i+1);
    }
\end{lstlisting}
\verb|i| gets cast to n immediately on LHS of definition.

\begin{lstlisting}[title={Transpose of a Product}]
f: T<n,m> -> T<m,k> -> T<n,k>;
f a b = c; 
    { c : T<n,k>;
      c[i,j] = d[j,i];
        { d : T<k,n>;
          d = a[j,l] * b[l,i];
        }
    } 
\end{lstlisting}
Binding of \verb|j|, \verb|l|, and \verb|i| to their appropriate 
dimensions is deferred until evaluation of innermost scope. 
\par One can also use \verb|_| to set all entries to some 
constant without the need to bind another variable. For example,
\begin{lstlisting}
t : T<n>;
t[_] = 0;
\end{lstlisting}
