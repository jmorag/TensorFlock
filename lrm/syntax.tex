\section{Syntax}%
\label{sec:syntax}
\subsection{Comments}
Comments in TensorFlock generally follow the C-style convention. Single line
comments are introduced with two forward slashes: \lstinline|//|. Multi-line
comments are introduced with \lstinline|/*| and conclude with \lstinline|*/|.
TensorFlock does not support nested multiline comments.
\subsection{Identifiers}
Identifiers in TensorFlock consist of a single lower case letter; followed by zero 
or more letters of either case, numbers, or underbar \lstinline|_|.
Identifiers may also conclude with zero or more single apostrophes \lstinline|'| .
\subsection{Keywords}
Keywords in TensorFlock consist of primitive types and conditional control flow:
\begin{lstlisting}
Int
Bool
T
If
Then
Else
True
False
\end{lstlisting}
\subsection{Scope}
Scope in TensorFlock is indicated with the use of \lstinline|{}|. The opening brace can be considered analogous to a \lstinline|where| statement.
\begin{lstlisting}
// Scope usage
scope : T<>;
scope = n; 
{ 
    n: T<>; n = 4.12;
    q : T<>; q = 12.; 
    p : Int; p = 20;
}
\end{lstlisting}
Functions defined inside of the curly braces are visible to the right hand
side of their corresponding function's definition and to everything defined
within their own scope. TensorFlock does not support name shadowing of any
kind. If a function name is used twice, the compiler will complain. Should one
feel especially inclined to reuse a name, the recommended style to simulate
this is to use the name followed by an apostrophe, and to add an apostrophe
after every subsequent attempt at shadowing. Chances are, if one accumulates
more than two or three apostrophes, there is a design flaw in the program.

\begin{lstlisting}
shadowedScope : Int;
shadowedScope = a; {
    a : Int; a = b + c; {
        b : Int; b = 2;
        c : Int; c = a; { 
            a : Int; a = 2; //Error!
        }
    }
}
\end{lstlisting}

\begin{lstlisting}
nestedScope : Bool;
nestedScope = x != y;
{// this is an example of properly nested scope that contains no shadowing
 // and will therefore compile.
    x : T<>; x = j*k;
    { 
        j : T<>; j = 2.;
        k : T<>; k = 3.;
    }
    y : T<>; y = 7.;
}    
\end{lstlisting}

\subsection{Primitives}
Primitives consist of a declaration, followed by a definition on the subsequent line. 
Declarations are composed of an identifier, a semicolon, and then type. 
Definitions consist of the same identifier, the assignment operator \lstinline|=|, 
and an expression. For example:
\begin{lstlisting}
anInt : Int;
anInt = 3;

aDouble : T<>;
aDouble = 3.14

aBoolean : Bool;
aBoolean = True;
\end{lstlisting}

N-dimensional tensors are the primary feature of TensorFlock, and as such, 
are considered a primitive in the language. Like above, tensors are declared 
with their dimensions, and then defined. Tensor components can only consist of
floating point values. 
\begin{lstlisting}
aTensor : T<2, 2, 2>
aTensor = [[[1.0, 2.0], [3.0, 4.0]], [[5.0, 6.0], [7.0, 8.0]]];
\end{lstlisting}
Tensor components are arranged in row-major order may be indexed using square
brackets and commas.
\begin{lstlisting}
someDouble : T<>;
someDouble = aTensor[0,1,1]; // == 4.0
\end{lstlisting}
\subsection{Function Declaration and Definition}
Functions are introduced with a single line declaration followed by a function definition of 
one or more lines. Declarations are composed of a function identifier, a semicolon, 
and then type definitions for its arguments and return value.
\begin{lstlisting}
add : Int -> Int -> Int;
add x y = x + y;
\end{lstlisting}
\subsection{Function Application}
Function application consists of juxtaposition of a function identifier and its 
subsequent arguments.
\begin{lstlisting}
add : Int -> Int -> Int;
add x y = x + y;

a : Int;
a = add 2 3;
\end{lstlisting}

Partial application of functions is not supported. If fewer than the specified number of 
arguments are provided, the compiler will return an error. Additionally TensorFlock 
does not support higher-order functions. Functions cannot be passed to
other functions as values nor can they be returned from other functions. While
this may seem like a large limitation for a functional language, most, if not
all of the standard map or fold operations that one could think of applying to
tensors can be implemented by other means in TensorFlock.
\subsection{Grammar}
\begin{tt}
\textit{
\begin{center}
\begin{longtable}{r c l}
program & $\rightarrow$ & declaration {\bf EOF }\\
& & \\
declaration & $\rightarrow$ & {\bf NONE} \\
 & \emph{|} & declaration \\
 & \emph{|} & function \\
 & & \\
function & $\rightarrow$ & function-type\emph{;} function-definition\emph{;} \\
 & & \\
function-type & $\rightarrow$ & id\emph{:}types\emph{;} \\
 & & \\
types & $\rightarrow$ & type\textsubscript{1}, \ldots, type\textsubscript{n}\emph{;} \\
 & & \\
function-definition & $\rightarrow$ & id formal-parameters \emph{=} expr\emph{;} \{ declaration \} \\
 & & \\
formal-parameters & $\rightarrow$ & {\bf NONE } \\
 & \emph{|} & id\textsubscript{1}, \ldots, id\textsubscript{n} \\
 & & \\
types & $\rightarrow$ & type\textsubscript{1}, \ldots, type\textsubscript{n} \\
 & & \\
type & $\rightarrow$ & \emph{Int} \\
 & \emph{|} & \emph{Bool} \\
 & \emph{|} & \emph{T<}shape-expr\textsubscript{1}, \ldots, 
              shape-expr\textsubscript{n}\emph{>} \\
 & & \\
expr & $\rightarrow$ & \emph{if} expr \emph{then} expr \emph{else} expr \\
& \emph{|} & expr \emph{||} expr \\
& \emph{|} & expr \emph{\&\&} expr \\
& \emph{|} & expr \emph{==} expr \\
& \emph{|} & expr \emph{!=} expr \\
& \emph{|} & expr \emph{<} expr \\
& \emph{|} & expr \emph{<=} expr \\
& \emph{|} & expr \emph{>} expr \\
& \emph{|} & expr \emph{>=} expr \\
& \emph{|} & expr \emph{+} expr \\
& \emph{|} & expr \emph{-} expr \\
& \emph{|} & expr \emph{*} expr \\
& \emph{|} & expr \emph{/} expr \\
& \emph{|} & expr \emph{\%} expr \\
& \emph{|} & \emph{-} expr \\
& \emph{|} & expr \emph{\^} expr \\
& \emph{|} & expr expr \\
& \emph{|} & tensor-id\emph{[}expr\textsubscript{1}, \ldots, expr\textsubscript{n}\emph{]} \\
& \emph{|} & literal \\
& & \\
literal & $\rightarrow$ &  integer-literal \\
& \emph{|} & floating-literal \\
& \emph{|} & boolean-literal \\
& \emph{|} & \emph{[}tensor-literal-list\emph{]} \\
& & \\
tensor-literal-list & $\rightarrow$ & 
        literal\\
& \emph{|} & tensor-literal-list, literal\\
& & \\
shape-expr & $\rightarrow$ & shape-expr \emph{+} shape-expr \\
& \emph{|} & shape-expr \emph{-} shape-expr \\
& \emph{|} & shape-expr \emph{*} shape-expr \\
& \emph{|} & shape-expr \emph{/} shape-expr \\
& \emph{|} & shape-expr \emph{\%} shape-expr \\
& \emph{|} & \emph{-}shape-expr \\
& \emph{|} & shape-expr \emph{\^} shape-expr \\
& \emph{|} & shape-function-expression \\
& \emph{|} & shape-function-id shape-expr\textsubscript{1}, \ldots, shape-expr\textsubscript{n} \\
& \emph{|} & shape-literal \\
& & \\
shape-literal & $\rightarrow$ & integer-literal \\
& \emph{|} & id \\
\end{longtable}
\end{center}
}
\end{tt}

